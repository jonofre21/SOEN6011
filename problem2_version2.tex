%%%%%%%%%%%%%%%%%%%%%%%%%%%%%%%%%%%%%%%%%%%
%%% DOCUMENT PREAMBLE %%%
\documentclass[12pt]{report}
\usepackage[english]{babel}
%\usepackage{natbib}
\usepackage{url}
\usepackage[utf8x]{inputenc}
\usepackage{amsmath}
\usepackage{graphicx}
\graphicspath{{images/}}
\usepackage{parskip}
\usepackage{fancyhdr}
\usepackage{vmargin}
\setmarginsrb{3 cm}{2.5 cm}{3 cm}{2.5 cm}{1 cm}{1.5 cm}{1 cm}{1.5 cm}
\title{Problem 2 Express Requirements Function $sinh(x)$ }							
% Title
\author{Jesus Onofre Diaz}						
% Author
\date{July 12, 2019}
% Date

\makeatletter
\let\thetitle\@title
\let\theauthor\@author
\let\thedate\@date
\makeatother

\pagestyle{fancy}
\fancyhf{}
\rhead{\theauthor}
\lhead{\thetitle}
\cfoot{\thepage}
%%%%%%%%%%%%%%%%%%%%%%%%%%%%%%%%%%%%%%%%%%%%
\begin{document}

%%%%%%%%%%%%%%%%%%%%%%%%%%%%%%%%%%%%%%%%%%%%%%%%%%%%%%%%%%%%%%%%%%%%%%%%%%%%%%%%%%%%%%%%%

\begin{titlepage}
	\centering
    \vspace*{0.5 cm}
   % \includegraphics[scale = 0.075]{bsulogo.png}\\[1.0 cm]	% University Logo
\begin{center}    \textsc{\Large   SOEN 6011}\\[2.0 cm]	\end{center}% University Name
	\textsc{\Large Software Processes }\\[0.5 cm]				% Course Code
	\rule{\linewidth}{0.2 mm} \\[0.4 cm]
	{ \huge \bfseries \thetitle}\\
	\rule{\linewidth}{0.2 mm} \\[1.5 cm]
	
	\begin{minipage}{0.4\textwidth}
		\begin{flushleft} \large
		%	\emph{Submitted To:}\\
		%	Name\\
          % Affiliation\\
           %contact info\\
			\end{flushleft}
			\end{minipage}~
			\begin{minipage}{0.4\textwidth}
            
			\begin{flushright} \large
			\emph{Submitted By : 40035954} \\
		Jesus Onofre Diaz
		https://github.com/jonofre21
		\break
			July 12, 2019
		\end{flushright}
	
           
	\end{minipage}\\[2 cm]
	
%	\includegraphics[scale = 0.4]{Con_Logo1.jpg}

\end{titlepage}

%%%%%%%%%%%%%%%%%%%%%%%%%%%%%%%%%%%%%%%%%%%%%%%%%%%%%%%%%%%%%%%%%%%%%%%%%%%%%%%%%%%%%%%%%

\tableofcontents
\pagebreak

%%%%%%%%%%%%%%%%%%%%%%%%%%%%%%%%%%%%%%%%%%%%%%%%%%%%%%%%%%%%%%%%%%%%%%%%%%%%%%%%%%%%%%%%%
\renewcommand{\thesection}{\arabic{section}}
\section{Function Express Requirements}
 

This report aims to describe the $y=sinh(x)$ function requirements and assumptions for a calculator application according to the ISO/IEC/IEEE 29148 standards.
This function is expressed by the formula:
$sinh(x)= \frac{e^x- e^{-x}}{2}$ where e is constant value.

\subsection{Functional Assumptions}

\begin{itemize}
\item Assumption 1
\begin{itemize}
\item ID:       FUNA1
\item Version:  1.0
\item Type:     functional
\item Owner:    Jesus
\item PRIORITY: 1 
\item Difficulty: Easy
\item DESC: Floating-point values are returned for floating-point arguments.
\item Rationale:when $x=2.8$  $y = 8.19$

\end{itemize}
\item Assumption 2
\begin{itemize}
\item ID:       FUNA2
\item Version:  1.0
\item Type:     functional
\item Owner:    Jesus
\item PRIORITY: 1 
\item Difficulty: Easy
\item DESC: $y=sinh(x) =0$ when x =0, this is due to the nature of the formula $sinh(x)= \frac{e^x- e^{-x}}{2} = \frac{1 - 1}{2}=0$ exact for finite x.
\item Rationale:$x=0$
\end{itemize}
\pagebreak

\item Assumption 3
\begin{itemize}
\item ID:       FUNA4
\item Version:  1.0
\item Type:     functional
\item Owner:    Jesus
\item PRIORITY: 1 
\item Difficulty: Easy
\item DESC: The sinh(x) curve is positive where $e^{x}$ is large, and negative where $e^{-x}$ is large.
\item Rationale: $e^{x}$ and $e^{-x}$
\end{itemize}
\end{itemize}

\pagebreak
\subsection{Requirements}

\begin{itemize}
\item Requirement 1
\begin{itemize}
\item ID:       FUNR1
\item Version:  1.0
\item Type:     functional
\item Owner:    Jesus
\item PRIORITY: 1 
\item Difficulty: Easy
\item DESC: The arguments passed to the function sinh(x) shall be real numbers from $-\propto to +\propto$ and they can be expressed in radians.
\item Rationale:when $x=2.2$, $ y = 4.45$ where x expressed in radians.
\end{itemize}
\item Requirement 2
\begin{itemize}
\item ID:       FUNR2
\item Version:  1.0
\item Type:     functional
\item Owner:    Jesus
\item PRIORITY: 1 
\item Difficulty: Easy
\item DESC: Euler's Number (e) is an irrational number which is a constant that has an approximate value of 2.71828, this value will be provided
\item Rationale: $(1 + 1/n)n$ where $n>0 $ and $n<=1000$
\end{itemize}
\pagebreak
\item Requirement 3
\begin{itemize}
\item ID:       FUNR3
\item Version:  1.0
\item Type:     functional
\item Owner:    Jesus
\item PRIORITY: 1 
\item Difficulty: Easy
\item DESC: Power function, this function will be calculated
\item Rationale:for an even number $a^{b}= (a^2)^b/2$  odd $a^{b}=a*(a^2)^b/2$
for negative number $a^{-n}$ = $1/a^{n}$
\end{itemize}
\item Requirement 4
\begin{itemize}
\item ID:       FUNR4
\item Version:  1.0
\item Type:     functional
\item Owner:    Jesus
\item PRIORITY: 1 
\item Difficulty: Easy
\item DESC: Absolute value function, this function will be calculated
\item Rationale:for any real number $|−n| = n$  and for $|0| = 0$
\end{itemize}
\end{itemize}
%\begin{enumerate}
 %   \item qqc
  %  \item qqn
%\end{enumerate}
%This is how to describe the requirements
%[Condition][Subject][Action][Object][Constraint of Action]
%Example:
%

\newpage
\section{Acknowledgments}

 Professor P. Kamthan and his great group of teaching assistants for the material and guidance.
\newpage
 
\begin{thebibliography}{111}
   
  \bibitem{ACAMP}
Encyclopedia Britannica. (2019). Hyperbolic functions | mathematics. [online] Available at: https://www.britannica.com/science/hyperbolic-functions [Accessed 12 Jul. 2019].

%if the "underfill \hbox" warning bothers you uncomment the following line
%\raggedright
\bibitem{ACAMP}
    Hunsicker, E. (2019). http://www.mathcentre.ac.uk/. [online] Mathcentre.ac.uk. Available at: http://www.mathcentre.ac.uk/resources/workbooks/mathcentre/hyperbolicfunctions.pdf [Accessed 12 Jul. 2019]..
  
\bibitem{ACAMP} 
   Calculadora conversor, las mejores calculadoras online. (2019). Calcular seno hiperbólico online - Sus propiedades, fórmulas y más!. [online] Available at: https://www.calculadoraconversor.com/seno-hiperbolico/ [Accessed 12 Jul. 2019].


\end{thebibliography}
\end{document}
